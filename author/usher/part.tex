%%%%%%%%%%%%%%%%%%%%%part.tex%%%%%%%%%%%%%%%%%%%%%%%%%%%%%%%%%%
% 
% sample part title
%
% Use this file as a template for your own input.
%
%%%%%%%%%%%%%%%%%%%%%%%% Springer %%%%%%%%%%%%%%%%%%%%%%%%%%

\begin{partbacktext}
\part{Mike Usher: Quantitative Symplectic Geometry}

There were three lectures:\\

\begin{enumerate}

    \item \href{#u1}{Day 1: Symplectic Embedding Obstructions and Constructions}

    The question of which subsets of $\mathbb{R}^{2n}$ embed symplectically into which others has turned out to be quite rich and has led to the development of many techniques over the past 40 years. In my first lecture, I will explain proofs of classic results of Gromov that give obstructions to symplectic squeezing and packing, and will contrast this with cases where an explicit construction allows one to give a non-obvious positive answer to a symplectic embedding question.

    \item \href{#u2}{Day 2: Capacities and Symplectic Homology}

    The second lecture will formally introduce the notion of a symplectic capacity, and will discuss two examples of these: the Hofer-Zehnder capacity based on periodic orbits of Hamiltonian systems, and the Floer-Hofer-Wysocki capacity based on symplectic homology.
    
    \item \href{#u3}{Day 3: Obstructing Embeddings Using Equivariant Symplectic Homology}

    The third lecture will explain how $S^1$-equivariant symplectic homology supplies additional restrictions on symplectic embeddings, both via a sequence of capacities coming from spectral invariants associated to various homology classes, and via chain-level information that vanishes in homology but can in some cases be used to show that two known embeddings are not symplectically isotopic.

\end{enumerate}

\end{partbacktext}