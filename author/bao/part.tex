%%%%%%%%%%%%%%%%%%%%%part.tex%%%%%%%%%%%%%%%%%%%%%%%%%%%%%%%%%%
% 
% sample part title
%
% Use this file as a template for your own input.
%
%%%%%%%%%%%%%%%%%%%%%%%% Springer %%%%%%%%%%%%%%%%%%%%%%%%%%

\begin{partbacktext}
\part{Erkao Bao: Introduction to Contact Homology}

There were three lectures:\\

\begin{enumerate}
    \item \href{#b1}{Day 1: Moduli Spaces of $J$-holomorphic Curves and Compactness}

    In this lecture, we begin with an introduction to basic contact geometry. We then introduce $J$-holomorphic curves as the gradient of the action functional. The focus will be on the moduli space of $J$-holomorphic curves, with a discussion on compactness. We will provide heuristic definitions of cylindrical contact homology and full contact homology.
    
    \item \href{#b2}{Day 2: Cylindrical Contact Homology in Dimension Three via Obstruction Bundle Gluing}

    This lecture addresses the transversality issues associated with the moduli space of $J$-holomorphic curves. We specifically focus on cylindrical contact homology in the 3-dimensional case. The lecture will cover the resolution of transversality issues using obstruction bundle gluing techniques.

    \item \href{#b3}{Day 3: Semi-Global Kuranishi Structure and Full Contact Homology}

    In this lecture, we introduce the semi-global Kuranishi structure. We explore its application in relation to obstruction bundle gluing, including computations of simple examples. The discussion will culminate in the rigorous definition of full contact homology, facilitated by the semi-global Kuranishi structure.

\end{enumerate}

\end{partbacktext}