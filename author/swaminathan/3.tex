\chapter{Global Kuranishi Charts: Construction}
\label{s3}
\abstract{Following Abouzaid–McLean–Smith 2021, we explain the construction of global Kuranishi charts for genus 0 Gromov–Witten moduli spaces and show that the outcome of the construction is unique up to equivalence. Time permitting, we will also briefly discuss how one can extend this construction to settings beyond genus 0 GW theory.}

\section{The AMS Trick}

Take $(X^{2n},\omega)$, $A\in H_2(X; \mathbb{Z})$, and $J$ an $\omega$-tame almost complex structure on $X$. This gives $\overline{\mathcal{M}}_0(X,A,J)$ which consists of $J$-holomorphic maps $u$ from trees of spheres $\Sigma$ to $X$ with $u_k[\Sigma]=A$.

\begin{theorem}
[Abouzaid, McLean, Smith, 2021]

$\overline{\mathcal{M}}_0(X,A,J)$ has a global Kuranishi chart, constructed often making some choices, but the resulting chart is unique up to equivalence.

\end{theorem}

\begin{proposition}

\[
\overline{\mathcal{M}}_0^*(\mathbb{CP}^d, d ) = \left\{ f: \Sigma \to \mathbb{CP}^d \mid f \text{ is a degree } d \text{ genus } 0 \text{ stable map and } f \text{ is non-degenerate} \right\}
\]

where non-degenerate means the image of $f$ is not contained in any hyperplane on $\mathbb{CP}^n$. Then there is a smooth quasi-projective variety of the expected dimension.

\end{proposition}

Here is an example of a non-degenerate function:

\begin{example}

\[\mathbb{CP}^1 \to \mathbb{CP}^d\]
\[[u,v]\mapsto [u^d; u^{d-1}v, ..., v^d]\]

\end{example}

The proposition gives rise to a nontrivial family

\[
\begin{tikzcd}
	{\mathcal{C}} & {\mathbb{CP}^d} \\
	{\overline{\mathcal{M}}_0^*(\mathbb{CP}^n, d)}
	\arrow[from=1-1, to=1-2]
	\arrow[from=1-1, to=2-1]
\end{tikzcd}
\]

where $\mathcal{C}$ is also a smooth, quasiprojective variety.

Every map $\mathcal{L}\to Z$ is induced with an embedding $Z\stackrel{[s_0:... ]}{\hookrightarrow} \mathbb{CP}^n$ where $s_0, ..., s_p \in H^0(Z; \mathcal{L})$ have no common zero.

Let's recall some basic definitions about ample line bundles.

\begin{definition}

$\mathcal{L}$ is \textbf{very ample} if $\exists s_0,...,s_n \in H^0(Z; \mathcal{L})$ such that we get an embedding $Z\stackrel{[s_0:...:s_n]}{\hookrightarrow} \mathbb{CP}^n$

\end{definition}

\begin{definition}

$ \mathcal{L}$ is \textbf{ample} if there exists $m>1$ such that $L^{\otimes m}$ is very ample.

\end{definition}

\begin{proposition}
    If $Z$ is a cone, then $\mathcal{L}$ is ample if and only if $\deg \mathcal{L}>0$ on each component of $Z$.
\end{proposition}

We can finally state the AMS trick:

\begin{lemma}
[AMS Trick]

Suppose $Z$ is a compact complex manifold $\mathcal{L}\to Z$ is an ample line bundle and $ \mathcal{E}\to Z$ is a holomorphic vector bundle, and endow everything with the Hermitian action. For $k\gg 1$, define
\[
W_k := \text{Im} \left( H^0(Z; \mathcal{E} \otimes \mathcal{L}^{\otimes k})\otimes_{\mathbb{C}} \overline{H^0(Z; \mathcal{L}^{\otimes k})} \stackrel{\langle \cdot, \cdot \rangle}{\longrightarrow} \Omega^0(Z, \mathcal{E})\right).
\]

As $k\to \infty$, $W_k$ provides an $L^2$ extension of $\Omega^0(Z, \mathcal{E})$, i.e. for all $\xi \in \Omega^0(Z, \mathcal{E}), \exists k\gg 1$ and $\eta \in W_k$ such that $\langle \xi, \eta \rangle_{L_2}\neq 0$.

\end{lemma}

\section{Construction}

\subsection{Line Bundles on $X$}

Approximate $\Omega$ by a symplectic form $\Omega$ which tames $J$ and satisfies $[\Omega] \in H^2(X; \mathbb{Q})$. Clear the denominator to get $[\Omega] \in H^2(X; \mathbb{Z})/\text{torsion}$ which implies that there exists a $C^\infty$ complex line bundle $L_\Omega \to X$ such that the first Chern class $c_1(L_\Omega)=[\Omega]$.

From Chern-Weil theory, we have the following lemma:

\begin{lemma}

There exists a Hermitian metric and a Hermitian connection $\nabla$ on $L_\Omega$ such that its curvature is $-2\pi i \Omega$.

\end{lemma}

Notation: $[\Omega]\cdot A=d$.

\subsection{Framed Genus 0 Curves }

Let's start with a (genus $0$) $J$-holomorphic stable $u:\Sigma\to X$. Then $u^*\mid_\Sigma \to \Sigma$ has a holomorphic structured generated by $(u^*\nabla)^{0,1}$. We know that $\int u^* \Omega \ge 0$ on each component of $\Sigma$ where $u$ is $J$-holomorphic and $\Omega$ tames $J$. We also know that $\int u^*\Omega >0$ on each unstable component. This implies that the line bundle $u^*L_\Omega$ has non-negative degree on each component $\Sigma$ and positive degree on each unstable component of $\Sigma$.

From last time, we know that
\[
\dim_{\mathbb{C}} H^0(\Sigma; u^*L_\Omega)=d+1, \\
H^1(\Sigma; u^* L_\Omega) = 0.
\]

Choose a basis $F=(f_0,...,f_d)$ of $H^0(\Sigma; u^*L_\Omega)$, called a \textbf{framing}. Consider the degree $d$ genus $0$ stable map
\[
\Sigma \stackrel{\Phi_F=[f_0:...:f_d]}{\longrightarrow} \mathbb{CP}^d
\]
Then $(\Sigma, \Phi_F) \in \overline{\mathcal{M}}_0^*(\mathbb{CP}^d, d)$. So we have

\[\begin{tikzcd}
	\Sigma & {\mathcal{C}} & {\mathbb{CP}^d} \\
	{(\Sigma, \Phi_F)} & {\overline{\mathcal{M}}_0^*(\mathbb{CP}^d, d)}
	\arrow["{{i_F}}", hook, from=1-1, to=1-2]
	\arrow[from=1-1, to=2-1]
	\arrow[from=1-2, to=1-3]
	\arrow[from=1-2, to=2-2]
	\arrow["\in"{description}, draw=none, from=2-1, to=2-2]
\end{tikzcd}\]

where $i_F$ is a embedding as a fiber.

We have
\[
H(\Sigma, u, F) = \left( \int_\Sigma \langle f_i, f_j \rangle u^*\Omega \right)_{0\le i, j \le d}
\]
is a Hermitian positive definite $(d+1)\times (d+1)$ matrix.

\begin{definition}

A \textbf{framed genus 0 curve in $X$} is a tuple $(\Sigma, u,F)$ where
\begin{enumerate}
\item $\Sigma$ is a nodal genus 0 curve.
\item $u: \Sigma \to X$ is a $C^\infty$ map in class $A$ such that $\int u^*\Omega \ge 0$ on each component of $\Sigma$ and $\int u^*\Omega > 0$ on each unstable component.
\item $F: (f_0,...,f_d)$ is a basis of $H^0(\Sigma; u^*L_\Omega)$ such that the matrix $H(\Sigma, u, F)$ is positive definite.
\end{enumerate}

\end{definition}

So $(\Sigma, u, F)\sim (\Sigma',u', F')$ if

\[\begin{tikzcd}
	\Sigma \\
	&& X \\
	{\Sigma'}
	\arrow["u", from=1-1, to=2-3]
	\arrow["\varphi"', from=1-1, to=3-1]
	\arrow["\cong", from=1-1, to=3-1]
	\arrow["{u'}"', from=3-1, to=2-3]
\end{tikzcd}\]
commutes.

\subsection{Achieving Transversality }

Choose
\begin{enumerate}
\item A relatively ample line bundle $\mathcal{L}$ on $\mathcal{C} \to \overline{\mathcal{M}}_0^* ( \mathbb{CP}^d, d)$ equipped with a Hermitian metric such that the natural $U(d+1)$ action on $\overline{\mathcal{M}}_0^*(\mathbb{CP}^d, d)$ lifts to an action of $\mathcal{L}$ and preserves the metric.
\item A $\mathbb{C}$-linear connection on $T^{*\text{ }0,1} \mathcal{C}$ which is an invariant under the $U(d+1)$ action.
\item A $\mathbb{C}$-linear connection on $TX$ (viewed as a $\mathbb{C}$-vector bundle on $J$)
\item A large integer $k\gg 1$.
\end{enumerate}

For more details, see [Horschi, Swaminathan, 2021, Section 2.1].

\begin{proposition}

The \textbf{thickening} $\mathcal{T}$ is the space of tuples $(\Sigma, u, F, \eta)$ where
\begin{enumerate}
\item $(\Sigma, u, F)$ is a framed genus $0$ curve in $X$
\item We have
    \[
    \eta \in H^0(\Sigma; u^*TX\otimes i_F^*(T^{* \text{ }0,1}\mathcal{C}\otimes \mathcal{L}^{\otimes k}))\otimes_{\mathbb{C}} \overline{H^0(\Sigma; i_F^* \mathcal{L}^{\otimes k})}
    \]
    satisfying the equation
    \[
    \overline{\partial}_J u + \langle \eta \rangle \circ di_F=0
    \]
\end{enumerate}

\end{proposition}

\begin{proposition}

The \textbf{obstruction bundle} $\mathcal{E}\to \mathcal{T}$ is a vector bundle whose fiber over $(\Sigma, u, F, \eta)$ is
\[
E_{(\Sigma, u, F)}\oplus \mathcal{H}_{d+1}
\]
where $\mathcal{H}_{d+1}$ is a space of $(d+1)\times (d+1)$ Hermitian matrices.

\end{proposition}

\begin{proposition}

The \textbf{obstruction section} is
\[
\mathcal{S}(\Sigma, u, F, \eta)=(\eta, \log \mathcal{H}(\Sigma, u, F)).
\]

\end{proposition}

\begin{proposition}

The \textbf{symmetry group} is
\[
G=U(d+1).
\]

\end{proposition}

The key point is that $\mathcal{T}$ is cut out transversally for $k\gg 1$ by the AMS trick.