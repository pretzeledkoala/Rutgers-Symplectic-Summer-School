%%%%%%%%%%%%%%%%%%%%%part.tex%%%%%%%%%%%%%%%%%%%%%%%%%%%%%%%%%%
% 
% sample part title
%
% Use this file as a template for your own input.
%
%%%%%%%%%%%%%%%%%%%%%%%% Springer %%%%%%%%%%%%%%%%%%%%%%%%%%

\begin{partbacktext}
\part{Mohan Swaminathan: Global Kuranishi Charts}

There were three lectures:\\

\begin{enumerate}

    \item \href{#s1}{Day 1: Local Structure of Holomorphic Curve Moduli Spaces}

    We discuss how the implicit function theorem and gluing analysis give rise to ‘local Kuranishi charts’ for holomorphic curve moduli spaces. We also explain what it means for two or more such local Kuranishi charts to be ‘compatible’ on their overlap and briefly discuss how an atlas of Kuranishi charts allows us to ‘virtually count’ points in a compact moduli space of expected dimension 0.
    
    \item \href{#s2}{Day 2: Global Kuranishi Charts: Definitions and Preliminaries}

    We introduce the notion of a ‘global Kuranishi chart’ and explain how having one of these substantially simplifies the previous discussion. We also explain what it means for two global Kuranishi charts to be ‘equivalent’, which is analogous to the notion of compatibility for local Kuranishi charts. For the remainder, we discuss some geometric preliminaries necessary to understand the construction of global Kuranishi charts for moduli spaces of closed holomorphic curves of genus 0.

    \item \href{#s3}{Day 3: Global Kuranishi Charts: Construction}

    Following Abouzaid–McLean–Smith 2021, we explain the construction of global Kuranishi charts for genus 0 Gromov–Witten moduli spaces and show that the outcome of the construction is unique up to equivalence. Time permitting, we will also briefly discuss how one can extend this construction to settings beyond genus 0 GW theory.

\end{enumerate}

\end{partbacktext}