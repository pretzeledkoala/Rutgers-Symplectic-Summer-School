\chapter{Local Structure of Holomorphic Curve Moduli Spaces}
\label{s1}
\abstract{We discuss how the implicit function theorem and gluing analysis give rise to ‘local Kuranishi charts’ for holomorphic curve moduli spaces. We also explain what it means for two or more such local Kuranishi charts to be ‘compatible’ on their overlap and briefly discuss how an atlas of Kuranishi charts allows us to ‘virtually count’ points in a compact moduli space of expected dimension 0.}
\section{The Moduli Space}

Suppose $(X^{2n}, \omega)$ is a closed symplectic manifold, $A\in H_2(X; \mathbb{Z})$ with $m \ge 0$, and $J$ is an almost complex structure $X$ tamed by $\omega$, i.e. $\omega(v,Jv)>0$ where $0\neq v \in T_X X$. Let $U$ be a $J$-holomorphic map $\mathbb{CP}^1 \to X$. Here, $du$ is linear and $u_*[\mathbb{CP}^1]=A$. This gives the moduli space $\mathcal{M}_{0,m}(X, A, J)$.

Instead of just consider $\mathbb{CP}^1$, we can also consider more complicated objects where we glue together several $\mathbb{CP}^1$ components at nodes, giving a tree of $\mathbb{CP}^1$. The curves arising from $\text{trees of }\mathbb{CP}^1 \to X$ are called \textbf{nodal genus 0 curves}.

These nodal genus 0 curves allow us to enrich our moduli space. In particular, we can now study
\[
\overline{\mathcal{M}}_{0,m} (X,A,J) =\left\{ \left(\sum x_1,...,x_m, u: \Sigma \to X\right) | \text{condition} \right\}/\sim
\]
where the condition is that $\Sigma$ is a nodal genus 0 curve with $m$ marked points $x_1,...,x_m$, $u$ is J-holomorphic, $u_*[\Sigma]=A$ and $\# \text{Aut}(\Sigma, x_1,...,x_m,u) < \infty$. This condition is equivalent to the condition that on any component of $\Sigma$ where $u$ is constant, there are 3 special points.

We present two basic properties of this moduli space:

\begin{theorem}
[Gromov Compactness Theorem]

$\overline{\mathcal{M}}_{0,m} (X,A,J)$ is compact and Hausdorff.

\end{theorem}

\begin{theorem}

$\overline{\mathcal{M}}_{0,m} (X,A,J)$ has an expected/virtual dimension
\[
d=2(m-3+n+c_1(TX)\cdot A).
\]

\end{theorem}

Here is the motivating problem:

\begin{problem}

Can we count the number of points in $\overline{\mathcal{M}}_{0,m} (X,A,J)$ after cutting down the virtual dimension to $0$?

\end{problem}

The issue is that in general, the moduli space is not a manifold of the expected dimension.

\section{The Transverse Case}

\begin{problem}

Suppose that we are given $(\Sigma, x_1,...,x_m, u: \Sigma \to X) \in \overline{\mathcal{M}}_{0,m} (X,A,J)$? What is the local structure of $\overline{\mathcal{M}}_{0,m} (X,A,J)$ near this point?

\end{problem}

We have two cases: when $\Sigma$ is smooth and when $\Sigma$ is nodal.

\subsection{$\Sigma$ is Smooth}

Here, smooth means $\Sigma \cong \mathbb{CP}^1$. Recall that
\begin{itemize}
\item $\mathcal{B}=C^\infty (\Sigma, X)_A=\left\{ v:\Sigma \to X | v \text{ is }C^\infty, v_*[\Sigma]=A \right\}$, which we should think of an infinite dimensional manifold.
\item $\mathcal{E}$ is an infinite rank vector bundle on $\mathcal{B}$ whose fiber is $\Omega^{0,1}(\Sigma, v^*TX) = \overline{\text{Hom}}_{\mathbb{C}}(T\Sigma, v^*TX)$ over any $v\in  g\mathcal{B}$.
\item $\sigma$ is a section of $\mathcal{E}$ over $\mathcal{B}$ given by $\left(v\mapsto \overline{\partial}_J v\right)$ where $\overline{\partial}_K = \frac{1}{2} (dv+J(v)\cdot dv \cdot j_\Sigma)$
\end{itemize}

Note that $\sigma^{-1}(0)=\text{Hol}(\Sigma, X,A,J)$.

We know that $u \in \sigma^{-1}(0)$, which means $\sigma$ has a well-defined linearization of $u$, namely
\[
D_u \sigma: T_u \mathcal{B} \to \mathcal{E}_u.
\]
More explicitly, this map is given by
\[
D(\overline{\partial}_J)_u \sigma: \Omega^0(\Sigma, u^*TX) \to \Omega^{0,1}(\Sigma, u^* TX).
\]

In holomorphic local coordinates, $z=s+it$ on $\Sigma$,
\[
\overline{\partial}_J u = \dfrac{1}{2} \left(\dfrac{\partial U}{\partial s} + J(u) \dfrac{\partial u}{\partial t}\right)\otimes (ds-idt).
\]

Given $\xi \in \Omega^0(\Sigma, u^*TX)$, we want to look at $\frac{d}{d\epsilon}|_{\epsilon =0}(u+\epsilon \xi)$. We have
\[
D \left(\overline{\partial}_J\right)_u \xi = \dfrac{1}{2} \left(\dfrac{\partial \xi}{\partial s} + J(u)\dfrac{\partial \xi}{\partial t} + \left(\partial_\xi J\right)(u) \dfrac{\partial u}{\partial t}\right)\otimes (ds-idt).
\]
We can check that $\dfrac{\partial \xi}{\partial s} + J(u)\dfrac{\partial \xi}{\partial t}$ is 1st order and $\left(\partial_\xi J\right)(u) \dfrac{\partial u}{\partial t}$ is 0th order.

For notation purposes, write $D_u:= D(\overline{\partial}_J)_u$. We have $D_u$ is Fredholm, i.e. $\text{ker, coker}$ are finite dimensional, and
\[
\text{ind}(D_u)=\dim(\ker D_u)-\dim (\text{coker} D_u)= 2(n+c_1(TX)\cdot A)
\]
where the second equality comes from Riemann-Roch.

The implicit function theorem tells us that if $D_u$ is surjective, then $\mathcal{M}_{0,m}(X,A,J)$ is a orbifold of expected dimension near $(\Sigma, x_1,...,x_m, u)$.

We conclude that
\[
\mathcal{M}_{0,m}(X,A,J) = \text{Hol}(\mathbb{CP}^1, X,A, J)\times \left(\left(\mathbb{CP}^1\right)^m \backslash \Delta\right)/\text{PSL}_2(\mathbb{C})
\]
where $\Delta = \left\{ (x_1,...,x_m)|x_i=x_j \text{ for some }i\neq j\right\}$.

\subsection{$\Sigma$ is Nodal}

Here, nodal means $\Sigma \cong$ trees of $\mathbb{CP}^1$'s. Consider $\tilde{\Sigma}$, the normalization of $\Sigma$ where we disjoint the spheres. The map
\[
\tilde{u}: \tilde{\Sigma} \stackrel{\text{normalization}}{\longrightarrow} \Sigma \stackrel{u}{\to} \Sigma
\]
gives
\[
D_{\tilde{u}}: \Omega^0  \left(\tilde{\Sigma}, \tilde{u}^* TX\right)\to \Omega^{0,1}(\tilde{\Sigma}, \tilde{u}^*TX)
\]
and
\[
D_{{u}}: \Omega^0  \left({\Sigma}, {u}^* TX\right)\to \Omega^{0,1}(\tilde{\Sigma}, \tilde{u}^*TX)
\]
where $\Omega^0(\Sigma, u^*TX) \subseteq \Omega^0  \left(\tilde{\Sigma}, \tilde{u}^* TX\right)$.

\begin{exercise}

Check that $\text{ind}(D_u)=2(n+c_1(TX)\cdot A)$

\end{exercise}

\begin{theorem}
[Gluing Theorem]

If $D_u$ is surjective, then $\overline{\mathcal{M}}_{0,m}(X,A,J)$ has a local chart near $(\Sigma,x_1,...,x_m, u)$ of the form $V/\Gamma$ where $V$ is a vector space of the expected dimension and $\Gamma$ is a finite group acting linearly on the vector space.

\end{theorem}

\section{Local Kuranishi Charts}

Let $\mathcal{B}$ be a Banach manifold, $\mathcal{E}$ a Banach vector bundle, and $J$ a smooth section with Fredholm linearizations. We want to study $\overline{\mathcal{M}} = \left(D \overline{\partial}\right)^{-1}(0)\subset \mathcal{B}$.

Suppose $u\in \overline{\mathcal{M}}$ is given. If $\left(D \overline{\partial}\right)_u$ is surjective, then we are done, i.e. $\overline{\mathcal{M}}$ is a manifold near $u$ and $T_u \overline{\mathcal{M}}=\ker \left(D \overline{\partial}\right)_u$.

So let's assume that $D_u: T_u \mathcal{B}\to \mathcal{E}_u$ is not surjective. Choose a finite dimensional vector space $V$ and a linear map $\lambda: E \to \mathcal{E}_u$ such that $E\twoheadrightarrow \text{coker}D_u$, i.e. $D_u \oplus \lambda: T_u \mathcal{B} \oplus E \twoheadrightarrow \mathcal{E}_u$. Choose a neighborhood $u\in \mathcal{U} \subset \mathcal{B}$ and an extension $\lambda: \mathcal{U}\times v \to \mathcal{E}_{\mathcal{U}}$.

Now, consider
\[
\mathcal{M}_{\mathcal{U}, E, \lambda} = \left\{ v\in \mathcal{U}, e\in E | \overline{\partial}v+\lambda(v,e) =0 \right \} \hookleftarrow  \overline{\mathcal{M}}\cap \mathcal{U}.
\]
There is a projection $s: \overline{\mathcal{M}}_{\mathcal{U}, E, \lambda}\to E$.

Let's consider the linearized operators at $(u,0) \in \overline{\mathcal{M}}_{\mathcal{U}, E, \lambda}$ given by
\[
T_u \mathcal{B} \to \mathcal{E}_u \\
(\xi, e)\mapsto D_u \xi + \lambda(u,e)
\]
which is surjective.

\begin{definition}

Suppose $\overline{\mathcal{M}}$ is a compact Hausdorff space. Then a \textbf{local Kuranishi chart of virtual dimension $d$} for $\overline{\mathcal{M}}$ is a quintuple $(\overline{\mathcal{M}}, E_\alpha, \Gamma_\alpha, s_\alpha, \psi_\alpha)$ where we have
\begin{itemize}
\item A finite dimensional topological manifold $\overline{\mathcal{M}}_\alpha$
\item A finite dimensional vector space $E_\alpha$ such that $\dim \overline{\mathcal{M}}_\alpha = d+ \dim E_\alpha$
\item A finite group $\Gamma_\alpha$ which acts on $\overline{\mathcal{M}}_{\alpha}$ and $E_\alpha$
\item A $\Gamma_\alpha$-equivariant function $s_\alpha: \mathcal{M}_\alpha \to E_\alpha$
\item A homeomorphism $\psi_{\alpha}:s_\alpha^{-1}(0)/\Gamma_\alpha \stackrel{\cong}{\to} U_\alpha \subset \overline{\mathcal{M}}$ where the subset is open.
\end{itemize}

\end{definition}

The upshot is that $\overline{\mathcal{M}}_{0,m}(X,A,J)$ is covered by local Kuranishi charts.

A local Kuranishi chart $(\overline{\mathcal{M}}, E_\alpha, \Gamma_\alpha, s_\alpha, \psi_\alpha)$ for $\overline{\mathcal{M}}$ induces a local virtual fundamental class on $U_\alpha$ via the following map:

\begin{align*}
\widecheck{H}_c^d (U_\alpha; \mathbb{Q}) &\overset{\frac{1}{|\Gamma_a|}\psi_\alpha^*}{\underset{\cong}{\longrightarrow}} \widecheck{H}_c^d \left(s_a^{-1}(0); \mathbb{Q} \right)^{\Gamma_\alpha} \\
& \overset{\text{Pardon}}{\underset{\cong}{\longrightarrow}} H_{\dim E_\alpha} \left(\overline{\mathcal{M}}_\alpha, \overline{\mathcal{M}}\backslash s_\alpha^{-1}; \mathbb{Q}\right)^{\Gamma_\alpha} \\
&\stackrel{(s_\alpha)_*}{\longrightarrow} H_{\dim E_\alpha} \left(E_\alpha, E_\alpha\backslash s_\alpha^{-1}; \mathbb{Q}\right)^{\Gamma_\alpha} \\
&\overset{\text{orientation}}{\underset{\cong}{\longrightarrow}} \mathbb{Q}
\end{align*}

where Pardon is the map found in \text{[Pardon, 2016, Appendix A]}.

\begin{definition}

The \textbf{local virtual fundamental class} is this entire map
\[
[v_\alpha]_{\text{local}}^{\text{vir}}: \widecheck{H}_c^d (U_\alpha; \mathbb{Q})\to \mathbb{Q}.
\]

\end{definition}

\begin{example}

For $\overline{\mathcal{M}}_\alpha = \mathbb{C}, E_\alpha = \mathbb{C}, \Gamma_\alpha=(1), s_\alpha(z)=z^n$, then
\[
[v_\alpha]_{\text{local}}^{\text{vir}}=n[\text{pt}].
\]

\end{example}
