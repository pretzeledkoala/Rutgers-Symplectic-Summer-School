\chapter{Global Kuranishi Charts: Definitions and Preliminaries}
\label{s2}
\abstract{We introduce the notion of a ‘global Kuranishi chart’ and explain how having one of these substantially simplifies the previous discussion. We also explain what it means for two global Kuranishi charts to be ‘equivalent’, which is analogous to the notion of compatibility for local Kuranishi charts. For the remainder, we discuss some geometric preliminaries necessary to understand the construction of global Kuranishi charts for moduli spaces of closed holomorphic curves of genus 0.}
\section{Global Kuranishi Charts and Equivalence}

\begin{definition}

Given $\overline{\mathcal{M}}$ a compact Hausdorff space, a \textbf{global Kuranishi chart of virtual dimension $d$} for $\overline{\mathcal{M}}$ consists of
\begin{itemize}
\item A finite dimensional topological manifold $\mathcal{T}$, called the \textbf{thickening}
\item A finite rank vector bundle on $\mathcal{E}\to \mathcal{T}$, called the \textbf{obstruction bundle}
\item A section $S: \mathcal{T}\to \mathcal{E}$, called the \textbf{obstruction section}
\item A compact Lie group $G$ which acts on $s^{-1}: \mathcal{E}\to \mathcal{T}$ such that the action of $G$ on $\mathcal{T}$ has finite stabilizers (and these stabilizers act linearly in suitable local coordinates), satisfying $\dim \mathcal{T}= d+ \text{rank} \mathcal{E} - \dim G$, called the \textbf{symmetry group}
\item A homeomorphism $s^{-1}(0)/G \stackrel{\sim}{\longrightarrow} \overline{\mathcal{M}}$, called the \textbf{footprint map}
\end{itemize}

\end{definition}

Why do we allow infinite $G$? It turns out allowing finite groups is not flexible enough. For example, if we take $\mathbb{CP}^1$ and consider a disk $\mathbb{D}$ at the origin with a $\mathbb{Z}/2$ action, we need infinite $G$.

On the other hand, our current condition is sufficient: Every effective orbifold is a global quotient $\mathcal{M}/G$ for some orbifold $M$ quotiented by a Lie group $G$.

We want to study orientations on
\begin{itemize}
\item $\mathcal{T}, \mathcal{E}$ preserved by $G$
\item $\mathfrak{g}=\text{Lie}(G)$
\end{itemize}

which together induce a virtual fundamental class for $\overline{\mathcal{M}}$:

\begin{align*}
\widecheck{H}^*(\overline{\mathcal{M}}, \mathbb{Q}) &\stackrel{\text{Poincaré Duality}}{\longrightarrow}H_{\text{rank } \mathcal{E}} (\mathcal{T}/G; \mathcal{T}/g - \overline{\mathcal{M}}, \mathbb{Q}) = H_{\text{rank } \mathcal{E}}^G (\mathcal{T}- \mathcal{T}-s^{-1}(0); \mathbb{Q}) \\
&\stackrel{s_*}{\longrightarrow} H_{\text{rank }\mathcal{E}}^G (\mathcal{E}, \mathcal{E}- 0_{\mathcal{E}}; \mathbb{Q}) \\
&\stackrel{\tau_{\mathcal{E}}^G}{\longrightarrow} H_0^G(\text{pt}, \mathbb{Q})\cong \mathbb{Q}
\end{align*}

where the first $\longrightarrow$ follows from Poincaré duality.

\begin{problem}

When are two global Kuranishi charts equivalent?

\end{problem}

\begin{proposition}

Two global Kuranishi charts are equivalent if we can reach one from the other using a sequence consisting of the following moves:
\begin{enumerate}
\item Germ equivalence: choose a $G$-invariant neighborhood $s^{-1}(0) \subset \mathcal{U}\subset \mathcal{T}$ where the second subset is open, and take $(G, \mathcal{U}, \mathcal{E}|_{\mathcal{U}}, s|_{\mathcal{U}})$.
\item Group enlargement: choose another compact Lie group $H$ and a $G$-equivariant principle $H$-bundle $p:P\to \mathcal{T}$ and take $(G\times H, P, p^* \mathcal{E}, p^* s)$
\item Stabilization: choose a $G$-equivariant vector bundle $\pi: \mathcal{W}\to \mathcal{T}$ and take $(G, \mathcal{W}, \pi^*(\mathcal{E}\oplus \mathcal{W}), \pi^* s \oplus \Delta_{\mathcal{W}})$.
\end{enumerate}

\end{proposition}

This should be thought of as an analog of the Reidemeister moves.

\section{Complex Geometry Background}

On $\mathbb{CP}^1$, we have the tautological line bundle $\mathcal{O}(-1)\hookrightarrow \mathbb{CP}^n \times \mathbb{C}^{n+1}$, which has a dual $\mathcal{O}(1)$. Define $\mathcal{O}(k) = \mathcal{O}(1)^{\otimes k}$.

\subsection{Holomorphic Line Bundles on Curves (Riemann Surfaces)}

\begin{lemma}

Suppose $\Sigma$ is a Riemann surface, $L\to \Sigma$ is a $C^\infty$ complex line bundle, and $\nabla$ is a $\mathbb{C}$-linear connection on it. Then $\nabla^{0,1}$ defines an unique holomorphic structure on $L$.

\end{lemma}

\begin{proof}
Given $p\in \Sigma$, choose a $C^\infty$ section $\tau$ of $L$ defined near $p$ such that $\tau(p)\neq 0$. Then 

\[
\nabla^{0,1}(\tau) = g\otimes \tau
\]

where $g\in \Omega^{0,1}(\Sigma)$.

The $\overline{\partial}$-Poincaré lemma states that we can find a $C^\infty$-function such that $g=\overline{\partial}f$ near $p$. Define $\sigma = e^{-f}\tau$, and we have
\begin{align*}
\Delta^{0,1} \sigma &= e^{-f}\nabla^{0,1} \tau + \overline{\partial}(e^{-f})\otimes \tau \\
&= e^{-f}(g\otimes \tau - \overline{\partial}f \otimes \tau)\\
&= 0
\end{align*}

as desired.
\end{proof}

\begin{lemma}
Suppose $\Sigma$ is a nodal genus $0$ curve. Then the isomorphism class of a holomorphic line bundle $L$ on $\Sigma$ is determined by the degree of $L$ on each component of $\Sigma$.
\end{lemma}

\begin{corollary}

Consider $L\to \Sigma$ as above. Suppose $L$ has total degree $d$ and has degree $\ge 0$ on each component. Then $\dim H^0(\Sigma; L) = d+1$ and $\dim_{\mathbb{C}} H^1(\Sigma; L)=0$.

\end{corollary}

\subsection{Genus 0 Curves in $\mathbb{CP}^n$}

Suppose $X$ is a complex manifold with
\begin{enumerate}
\item a holomorphic map $f: X\to \mathbb{CP}^n$
\item a holomorphic line bundle $\mathcal{L}\to X$ and holomorphic sections $s_0,...,s_n$ such that they have no common zero in $X$.
\end{enumerate}

To go between these two, we can do the following:
\[
(X\stackrel{f}{\longrightarrow} \mathbb{P}^n_{[x_0:...:x_n]})\mapsto (f^* \mathcal{O}(1), f^*x_0,...,f^*x_n) \\
(\mathcal{L}, s_0,...,s_n) \mapsto (X\stackrel{[s_0:...:s_n]}{\longrightarrow} \mathbb{CP}^n ).
\]

Consider $\overline{\mathcal{M}}_{0,m}(\mathbb{CP}^n, d)$ for some $n, d\ge 1, m\ge 0$.

\begin{lemma}

This space is a complex orbifold of the expected dimension.

\end{lemma}

\begin{proof}
Take $f:\Sigma \to \mathbb{P}^n$ where $\Sigma$ is a genus $0$ nodal curve. Recall $D_f: \Omega^0(\Sigma, f^*T \mathbb{CP}^n)\to \Omega^{0,1}(\Sigma)$. We want to show that this is surjective, which is equivalent to $\text{coker}D_f = H^1(\Sigma; f^* T \mathbb{CP}^n )$. We have the Euler exact sequence:

\[
0 \longrightarrow \mathcal{O} \longrightarrow \mathcal{O}(1)^{\oplus n+1}\longrightarrow T \mathbb{CP}^n \longrightarrow
\]

If we pullback along $f:\Sigma \to \mathbb{CP}^n$ and take long exact sequence

\[
\longrightarrow H^1(\Sigma; f^* \mathcal{O})^{\oplus n+1}\longrightarrow H^1(\Sigma; f^*T \mathbb{CP}^n) \to 0
\]

and we are done.
\end{proof}