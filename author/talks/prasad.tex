\chapter{Rohil Prasad: High-Dimensional Families of Holomorphic Curves and Three-Dimensional Energy Surfaces}
\label{prasad}

\abstract{Let $H$ be any smooth function on $\mathbb{R}^4$. I’ll discuss some recent dynamical theorems for the Hamiltonian flow on level sets of H (“energy surfaces”). The results are proved using holomorphic curves and neck stretching. One important tool is the compactness theorem from Dan’s talk.}

\section{Basics}

Consider $(\mathbb{R}^n, \Omega = \sum_{i=1}^n \,dx_i \wedge \,dy_i)$, a Hamiltonian $H:\mathbb{R}^{2n} \to \mathbb{R}$, and the Hamiltonian vector field $X_H$.

\begin{lemma}

The flow of $X_H$ preserves $H$.

\end{lemma}

\begin{corollary}

The flow of $X_H$ preserves the level sets of $H$.

\end{corollary}

\begin{remark}

Historically, these $H$'s were called energy surfaces.

\end{remark}

\begin{definition}

Fix $s\in \text{Reg}(H)$. Then $ H^{-1}(s)$ is a smooth $(2n-1)$-dimensional manifold. Energy surfaces that arise this way are called \textbf{regular}.

\end{definition}

\section{Invariant Sets}

\begin{problem}
[Herman, 1998 ICM]

Fix $H:\mathbb{R}^{2n}\to \mathbb{R}$ and fix a compact regular energy surface. Does $Y$ contain a proper closed $X_H$ invariant subset?

\end{problem}

\begin{theorem}
[Fish, Hofer, 2018]

Yes for $n=2$.

\end{theorem}

Here is some more progress on the problem:

\begin{theorem}
[Weinstein, Rabinowitz, Vileda]

If $Y$ is of contact type, it contains a closed orbit.

\end{theorem}

\begin{theorem}
    Examples exist where $Y$ has no closed orbits:
        \begin{itemize}
            \item \text{[Ginzburg, Kerman, Herman]} for $n \geq 3$.
            \item \text{[Ginzburg, Gurel]} for $n \geq 2$, with $H$ being $C^2$.
        \end{itemize}
\end{theorem}

\begin{theorem}
[Prasad, 2024; Theorem A]

Let $H: \mathbb{R}^{n} \to \mathbb{R}$, $Y$ be a compact regular energy surface. There exists an infinite family of distinct, proper, closed invariant subsets with dense union in $Y$.

\end{theorem}

Notation: $\text{Reg}_C(J)=\{ s\in \text{Reg}(H)| H^{-1}(s)\text{ compact} \}$.

\begin{theorem}
[Theorem B]

Let $H': \mathbb{R}^{2n} \to \mathbb{R}$ for almost every $s\in \text{Reg}_c(H)$. $H'(s)$ has the following property: for any closed orbit $\Lambda \subset H^{-1}(s)$, $H^{-1}(s)/\Lambda$ is not minimal.

\end{theorem}

\begin{remark}

The Le Calvez-Yoccoz property implies dense existence on invariant sets.

\end{remark}

\section{Closed Orbits and Closed Holomorphic Curves}

\begin{theorem}
[Hofer, Zehnder, 1987]

Fix $H:\mathbb{R}^{2n} \to \mathbb{R}$. For about every energy surface $s\in \text{Reg}_c(H), H^{-1}(c)$ contains a closed orbit.

\end{theorem}

\begin{theorem}

Fix $H:\mathbb{R}^{2n} \to \mathbb{R}$. Almost every $s\in \text{Reg}_c H$, $H^{-1}(s)$ contains two closed orbits. This bound is sharp.

\end{theorem}

\begin{proof}
Consider
\[
H=\dfrac{x_1^2+y_1^2}{a}+\dfrac{x_2^2+y_2^2}{b}
\]
where $\frac{a}{b}\notin \mathbb{Q}$. Each regular energy surface has 2 closed orbits.
\end{proof}

\begin{theorem}

Fix $H:\mathbb{R}^4\to \mathbb{R}$. Under $C^\infty$-generic conditions on $H$, almost every compact regular energy surface has infinitely many closed orbits.

\end{theorem}

\begin{proof}
Follows from a strictly stronger version of Theorem B and known results about generic Hamiltonians.
\end{proof}

\begin{theorem}
[Taubes]

Fix $H: \mathbb{CP}^2 \to \mathbb{R}$. Fix base $J, J\ge 1, S\subset \mathbb{CP}^2$ such that $\# S \approx \lambda^2$. Then there exists a closed $J$-holomorphic curve $u: C\to \mathbb{CP}^2$ such that
\begin{enumerate}
\item $S=u(C)$
\item $\int_C u\Omega = d$
\item $\chi(C)\sim -d^2$
\end{enumerate}

\end{theorem}

\section{Theorem A Proof Idea}

Take
\[
W_k=\mathbb{CP}^2 \backslash Y \cup [-k,k]\times Y.
\]

Consider the sequence $u_k: C_k \to W_k$.

\begin{definition}

The \textbf{stretched limit set}
\[
\chi(u_k)\in \text{Cl}((-1,1)\times Y)\times (-1,1).
\]

We say $(\Xi, s)\in \chi(u_k)$ if there exists $\{s_k\}$ such that the following holds:

\begin{enumerate}
\item $u_k(C_k) \cap (s_k^{-1}, s_k +1) \times Y \to \Xi$ after shifting
\item $k^{-1}s_k \to s$.
\end{enumerate}

\end{definition}

Define $u_{d,k}$ be the set of degree $d$ curves as above. The main structural result is the following:

\begin{proposition}
\text{ }
\begin{enumerate}
\item For almost every $s$ and every $(\Xi, s) \in \chi(s_k)$,
    \[
    \Xi = (-1,1)\times \Lambda
    \]
    where $\Lambda$ is $X_H$ invariant.
\item For all but $\sim d^2$ heights $s$, every $(\Xi, s) \in \chi(u_{d,k})$ is such that $\Xi$ is $\epsilon$-almost invariant where $\epsilon\to 0$ as $d\to \infty$.
\end{enumerate}

\end{proposition}
