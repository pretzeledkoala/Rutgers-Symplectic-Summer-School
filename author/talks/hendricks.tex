\chapter{Kristen Hendricks: Symplectic Annular Khovanov Homology and Knot Symmetry}
\label{hendricks}
    
\abstract{Khovanov homology is a combinatorially-defined invariant which has proved to contain a wealth of geometric information. In 2006 Seidel and Smith introduced a candidate Lagrangian Floer analog of the theory, which has been shown by Abouzaid and Smith to be isomorphic to the original theory over fields of characteristic zero. The relationship between the theories is still unknown over other fields. In 2010 Seidel and Smith showed there is a spectral sequence relating the symplectic Khovanov homology of a two-periodic knot to the symplectic Khovanov homology of its quotient; in contrast, in 2018 Stoffregen and Zhang used the Khovanov homotopy type to show that there is a spectral sequence from the combinatorial Khovanov homology of a two-periodic knot to the annular Khovanov homology of its quotient. (An alternate proof of this result was subsequently given by Borodzik, Politarczyk, and Silvero.) These results necessarily use coefficients in the field of two elements. This inspired investigations of Mak and Seidel into an annular version of symplectic Khovanov homology, which they defined over characteristic zero. In this talk we introduce a new, conceptually straightforward, formulation of symplectic annular Khovanov homology, defined over any field. Using this theory, we show how to recover the Stoffregen-Zhang spectral sequence on the symplectic side. We further give an analog of recent results of Lipshitz and Sarkar for the Khovanov homology of strongly invertible knots. This is work in progress with Cheuk Yu Mak and Sriram Raghunath.}

\section{(Symplectic) Khovanov Homology}

Khovanov homology takes a link $L\subseteq S^3$ and gives back $\text{Kh}(L)$, a bigraded vector space over some field with $\chi$ the Jones polynomial.

\begin{theorem}
[Ozswatch, Szabé]

There exists a spectral sequence
\[
\widehat{\text{Kh}}(L; \mathbb{F}_2)\rightrightarrows \widehat{\text{HF}}(\Sigma(\overline{L}))
\]

\end{theorem}

Here is the Floer theory analogy:
\begin{itemize}
\item $(M,\omega)$ are exact, $\omega = d\lambda$, and convex at $\infty$
\item $L_0, L_1$ are exact compact Lagrangians satisfying $\lambda|_{L_i}=df_i$
\item $(M, L_0,L_1)\rightsquigarrow \text{CF}(M, L_0,L_1) = (\mathbb{F} \langle L_1 \text{ transverses }L_1 \rangle, \partial)$.
\end{itemize}

Let's briefly discuss symplectic Khovanov homology. Take $p = \prod_{i=1}^n (z - k_i)$ and consider:
\[
\left\{ u^2 + v^2 + p(z) = 0 : (u, v, z) \in \mathbb{C}^3 \right\}.
\]

Define
\[
\Sigma_{\beta_L}:= \left\{ (u, v, z) : z \in B_i, u, v \in \sqrt{-p(z)} \mathbb{R} \right\}.
\]

Let $\mathcal{Y}_n \subseteq \text{Hilb}^n(S) \xrightarrow{\text{Hilbert-Chow}} \text{Sym}^n(S)$ be 1-to-1 away from the diagonal, where $\text{Hilb}$ denotes the Hilbert scheme.

Define $\Sigma_A = \Sigma_{A_1} + \cdots + \Sigma_{A_n}$ and $\Sigma_B = \Sigma_{B_1} + \cdots + \Sigma_{B_n} \subseteq \text{Sym}^n(S) \setminus \Delta \subseteq \mathcal{Y}_n \subseteq \text{Hilb}^n(S)$, where
\[
\mathcal{Y}_n = \text{HC}^{-1}\left\{ (u_1, v_1, z_1), \ldots, (u_n, v_n, z_n) : z_i = z_j \implies (u_i, v_i) = (u_j, v_j) \right\}.
\]

Then, we have the following definition:

\begin{definition}

\[
\text{Kh}_{\text{symp}}(L) = \text{HF}(\Sigma_A, \Sigma_B).
\]

\end{definition}

\begin{theorem}
[Abouzaid, Smith]

Over characteristic 0, $\text{Kh}(K)=\text{Kh}_{\text{symp}}(K)$.

\end{theorem}

Since $O(2)$ acts on $(u,v)$, we have an action on $S$ and thus a symplectic action on $(\mathcal{Y}_n, \Sigma_A, \Sigma_B)$.

Let's briefly digress to discuss $\mathbb{F}_2$-actions:

\begin{example}
[Smith, Borel]

Consider $\tau$ acting on $X$ satisfying $\tau^2 = \text{Id}$, a fixed set $X^{\text{Fix}}$. Then there is a spectral sequence

\begin{align*}
H^*(x; \mathbb{F}_2)\otimes \mathbb{F}_2 \left[\theta, \theta^{-1}\right] &\rightrightarrows \theta^{-1} H_{\mathbb{Z}/2\mathbb{Z}}(X; \mathbb{F}_2) \\
& \simeq H^* (X^{\text{fix}}; \mathbb{F}_2)\otimes \mathbb{F}\left[\theta, \theta^{-1}\right]
\end{align*}

where
\[\mathbb{F}_2\left[\theta, \theta^{-1}\right] = \theta^{-1}H^*(\theta \mathbb{Z}_2; \mathbb{F}_2)
\]
and
\[
\theta^{-1} H_{\mathbb{Z}/2\mathbb{Z}}(X; \mathbb{F}_2) = \text{Ext}_{\mathbb{F}_2\left[\mathbb{Z}_2 \right]}(C_*(X), \mathbb{F}_2)
\]

\end{example}

\begin{example}
[Seidel, Smith, 2010]

Given $\tau$ acting on $(M,L_0, L_1)$ with $\tau^2 = \text{Id}, \tau^*\omega = \omega, \tau(L_i)=L_i$. Take $(M^{\text{Fix}},L_0^{\text{Fix}}, L_1^{\text{Fix}})$. Then
\[
\text{HF}(M, L_0, L_1)\otimes F_2[\theta, \theta^{-1}]\rightrightarrows \text{HF}(M^{\text{Fix}},L_0^{\text{Fix}}, L_1^{\text{Fix}})\otimes \mathbb{F}_2 [\theta, \theta^{-1}]
\]

\end{example}

The intrinsic symmetry is $(u,v,z)\to (u, -v,z)$.

\begin{theorem}
[Manelescu]

Consider

\[
\mathcal{Y}_n^{\text{Fix}}=\text{Sym}^n(\Sigma-\{ \vec{z} \})\backslash \nabla
\]
\[
\Sigma_A^{\text{Fix}}=\alpha_1\times ....\times \alpha_n
\]
\[
\Sigma_B^{\text{Fix}}=\beta_1\times ....\times \beta_n
\]

Then we have a spectral sequence
\[
\text{Kh}_{\text{symp}}(L)\otimes \mathbb{F}_2[\theta, \theta^{-1}]\rightrightarrows g\widehat{\text{HF}}(\Sigma (\overline{L})\otimes H_*(S^1))\otimes \mathbb{F}_2[\theta, \theta^{-1}].
\]

\end{theorem}

\subsection{Extrinsic Symmetries }

For a link $L$, we have $\mathcal{S}, \mathcal{Y}_n$. Do the same for $\overline{L}$. $\tau: (u,v,z)\mapsto (u,v,-z)$. $S/\tau =\overline{S}, z\mapsto z^2$.

\begin{theorem}

The following commutative diagram commutes:
\[\begin{tikzcd}
	{\text{Hilb}^{n/2}\left(\overline{S}\right)} && {\text{Hilb}^n(S)} \\
	\\
	{\text{Sym}^{n/2}\left(\overline{S}\right)} && {\text{Sym}^n(S)} \\
	{(u,v,z)} && {\left\{ (u,v,-\sqrt{z}), (u,v,-\sqrt{z})\right\}}
	\arrow[hook, from=1-1, to=1-3]
	\arrow["{\text{HC}}"', from=1-1, to=3-1]
	\arrow["{\text{HC}}", from=1-3, to=3-3]
	\arrow[hook, from=3-1, to=3-3]
	\arrow[maps to, from=4-1, to=4-3]
\end{tikzcd}\]

where $\tau$ acts on $\text{Hilb}^n(S)$ and $\text{Sym}^n(S)$.

\end{theorem}

\begin{theorem}
[Seidel, Smith]

\[
\text{Kh}_{\text{symp}}(L)\otimes \mathbb{F}_2[\theta, \theta^{-1}]\rightrightarrows \text{Kh}_{\text{symp}}(\overline{L})\otimes \mathbb{F}_2[\theta, \theta^{-1}]
\]

\end{theorem}

\begin{theorem}
[Stoffregen, Zhang, 2018; Borodzik, Politarcyzk, Silvero]

\[
\text{Kh}(L)\otimes \mathbb{F}_2[\theta, \theta^{-1}]\rightrightarrows \text{AKh}(\overline{L})\otimes \mathbb{F}[\theta, \theta^{-1}].
\]

\end{theorem}

\section{Annular (Symplectic) Khovanov Homology}

Following the works of [Asaeda; Przytycki, Sikora; Roberts].

$\text{AKh}$ is an associated graded ring of an annular filtration.

\begin{theorem}
[Mak, Smith, 2019]

Over characteristic $0$,
\[
\text{AKh}_{\text{symp}}^{HH}(L)\simeq \text{AKh}(L)
\]
where $HH$ is the Hochschild homology.

\end{theorem}

Now, we present a new $\text{AKh}_{\text{symp}}$. Replace $\text{Hilb}^n(S)$ with $\text{Hilb}^n(S\backslash \pi^{-1}(0))$, deleting a divisor over $0$.

\begin{theorem}
[Hendricks, Mak, Raghunath]

$\text{AKh}_{\text{symp}}(L)$ is a link invariant.

\end{theorem}

\begin{conjecture}

Over characteristic 0, $\text{AKh}_{\text{symp}}(L)$ is the same as $\text{AKh}_{\text{symp}}^{HH}(L)$.

\end{conjecture}

We have the following intrisic cases:
\begin{itemize}
\item $(u,v,z)\mapsto (u,-v,z)$:
\begin{align*}
\text{AKh}_{\text{symp}}(L)\otimes \mathbb{F}_2[\theta, \theta^{-1}] & \rightrightarrows g\widehat{CFK}(\Sigma(mL), \tilde{A})\otimes H_*(S^1) \otimes \mathbb{F}_2[\theta, \theta^{-1}]  \\
& \rightrightarrows \widehat{CFK}(\Sigma(mL), \tilde{A})\otimes H_*(S^1) \otimes \mathbb{F}_2[\theta, \theta^{-1}]
\end{align*}
\item $(u,v,z)\mapsto (u,v,-z)$
\[
\text{AKh}_{\text{symp}}(L)\otimes \mathbb{F}_2[\theta, \theta^{-1}]\rightrightarrows \text{AKh}_{\text{symp}}(\overline{L})\otimes \mathbb{F}_2[\theta, \theta^{-1}]
\]
\item $(u,v,z)\mapsto (-u,-v, -z)$: start with the original theory and do not delete a divisor.
\item $\{ (u,v,z), (-u, -v, -z)\}$:
\[
\text{Kh}_{\text{symp}}(L)\otimes \mathbb{F}_2[\theta, \theta^{-1}]\rightrightarrows \text{AKh}_{\text{symp}}(\overline{L})\otimes \mathbb{F}_2[\theta, \theta^{-1}].
\]
\end{itemize}

\section{The Table}
We can summarize the results in the following table:

\begin{tabular}{|c|c|c|}
\hline
\textbf{Action} & \textbf{Consequence} & \textbf{Analog} \\
\hline
$(u, v, z) \mapsto (u, -v, z)$ & $\text{Kh}_{\text{symp}}(L) \text{ to } \text{g}\widehat{\text{HF}}\left(\Sigma(mL)\right)$ & \text{[Ozsvath, Szabo]} \\
\hline
$(u, v, z) \mapsto (u, v, -z)$ & $\text{Kh}_{\text{symp}}(L) \text{ to } \text{Kh}_{\text{symp}}\left(\overline{L}\right)$ & \text{[None]} \\
\hline
$(u, v, z) \mapsto (u, -v, z)$ & $\text{AKh}_{\text{symp}}(L) \text{ to } \widehat{\text{HF}} \left( \Sigma(mL), \hat{A} \right)$ & \text{[Roberts]} \\
\hline
$(u, v, z) \mapsto (u, v, -z)$ & $\text{AKh}_{\text{symp}}(L) \text{ to } \text{AKh}_{\text{symp}}\left(\overline{L}\right)$ & \text{[Zhang]} \\
\hline
$(u, v, z) \mapsto (-u, -v, -z)$ & $\text{Kh}_{\text{symp}}(L) \text{ to } \text{AKh}_{\text{symp}}\left(\overline{L}\right)$ & \text{[Szabó, Ozsváth; Borodzik, Politarczyk, Silvero]} \\
\hline
\end{tabular}