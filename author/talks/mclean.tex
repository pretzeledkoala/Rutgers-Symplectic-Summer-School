\chapter{Mark McLean: Symplectic Orbifold Gromov-Witten Invariants}
\label{mclean}

\abstract{Chen and Ruan constructed symplectic orbifold Gromov-Witten invariants more than 20 years ago. In ongoing work with Alex Ritter, we show that moduli spaces of pseudo-holomorphic curves mapping to a symplectic orbifold admit global Kuranishi charts. This allows us to construct other types of Gromov-Witten invariants, such as K-theoretic counts. The construction relies on an orbifold embedding theorem of Ross and Thomas.}

\section{Introduction}

The aim is to construct $\partial W$ invariants for orbifolds. Over $\mathbb{Q}$, this was done by [Chen, Ruan]. We want to use global Kuranishi charts. This is part of a larger project proving a version of the crepant resolution conjecture. There is also work by [Mak, Seyfaddini, Smith] are working on the global quotient case. This talk is all joint work with [Ritter].

Informally, an orbifold is like a manifold, but the charts look like $V/\Gamma$ where $V\subset \mathbb{R}^n$ is an open finite subset, $\Gamma \to \text{GL}_n(\mathbb{R})$. For example, $\{\text{pt}\}/\mathbb{Z}/2$ is an orbifold. We will not define an orbifold formally.

Suppose $G$ is a compact Lie group acting on a smooth manifold $M$ with finite stabilizers. Then the quotient $[M/G]$ is naturally an orbifold.

\begin{theorem}
[The Slice Theorem]

For each point $x\in M$, there exists a $G_X$-equivariant submanifold $S_X \subseteq M$ containing $X$ and a $G$-equivariant neighborhood $U_X\subseteq M$ of $X$ such that
\[
G\times_{G_X} S_X\to U_X
\]
is a diffeomorphism.

\end{theorem}

\begin{definition}

If $S_X$ has a global $G_X$ equivariant coordinate system, then $(S_X, G_X)$ is an \textbf{orbifold chart} for $[M/G]$ centered at $X$.

\end{definition}

\begin{theorem}
[Pardon]

Every compact orbifold is of the form $[M/G]$.

\end{theorem}

\begin{definition}

Let $[M_1, G_1], [M_2, G_2]$ be orbifolds. A \textbf{Hilsum-Skandalis morphism} between these orbifolds is a diagram

\[\begin{tikzcd}
	P & {M_2} \\
	{M_1}
	\arrow["f", from=1-1, to=1-2]
	\arrow["\pi"', from=1-1, to=2-1]
\end{tikzcd}\]

where there is a $G_1\times G_2$ action on $P$, $\pi$ is a principal $G_2$-bundle that is $G_1$-equivariant, and $f$ is $G_2$-equivariant.

\end{definition}

We can define a symplectic orbifold, complex orbifold, etc.

\begin{definition}

If $X=[M/G]$ is an orbifold, then $\underline{X}=M/G$ is the \textbf{underlying coarse moduli space}.

\end{definition}


\section{Twisted Nodal Curves}

Let's fix $(X,\omega, J)$ where $X$ is an orbifold, $\omega$ is a symplectic form, and $J$ is a $J$-taming almost complex structure. Take $\beta \in H_2(\underline{X}; \mathbb{Z})$.

\begin{definition}

A \textbf{twisted nodal domain} $\Sigma$ is a space $\tilde{\Sigma}/\sim$ where $\tilde{\Sigma}$ is a 1d complex orbifold and the equivalence relation is identifies distinct pairs of points $p\sim q$ so that the following balancing condition holds:
\begin{itemize}
\item $p$ admits an orbifold chart with coordinate $z$ centered at $p$ and where the stabilizer group $\mathbb{Z}/k\mathbb{Z}$ acts by $(m,z)\to e^{2\pi i m} z$
\item $q$ admits an orbifold chart with coordinate $z$ centered at $q$ and where the stabilizer group $\mathbb{Z}/k\mathbb{Z}$ acts by $(m,w)\to e^{-2\pi i m} w$
\end{itemize}

\end{definition}

So, near a node, $\Sigma$ looks like $\{XY=0\}/\,\mathbb{Z}/k\mathbb{Z}$ with $(g,(x,y))=(gx,g^{-1}x)$. We can extend this to $\{XY=t\}/ \, \mathbb{Z}/k\mathbb{Z}$.

\begin{definition}

A \textbf{marking} on $\Sigma$ is a collection of distinct points $p_1,...,p_n$ disjoint from the nodes containing all smooth points with nontrivial stabilizer.

\end{definition}

To learn more, see [Abramovich, Vistoli].

\begin{definition}

A \textbf{twisted nodal curve} $U:\Sigma \to X$ is a $J$-holomorphic Hilsum-Scandalis morphism $\Sigma \to X$ such that
\begin{itemize}
\item this descends to a continuous map $\Sigma \to \underline{X}$
\item the induced map of stabilizer groups $G_\sigma\to G_{u(\sigma)}$ is injective for all $\sigma \in \tilde{\Sigma}$.
\end{itemize}

\end{definition}

[Abramovich, Vistoli] studied the example $\{\text{pt}\}/\,\mathbb{Z}/2=X$.

For smooth $g=0$ case, Sieburt considered $(u, \Sigma, F)$ where $u: \Sigma \to X$, $L\to X$, and the framing is a basis of $H^0(u^*L)$. Then, we obtain $\mathcal{E}\to \mathbb{P}^d$ where $d=\dim H^0(u^*L)-1$ and $\mathcal{F}=\mathcal{M}_{0,d}(\mathbb{P}^n)$.

\section{Problems}

There are many problems with generalizing this $g=0$ case to the smooth orbifold case:
\begin{enumerate}
\item For higher genus curves, there is a moduli space of line bundles in each given degree
\item Twisted nodal curves with nontrivial stabilizer groups don't map to $\mathbb{CP}^d$
\end{enumerate}

Let's deal with $(2)$, using an idea by [Ross, Thomas]. Instead of looking at curves mapping to $\mathbb{CP}^d$, we'll look at curves mapping to the weighted projective space
\[
P(w_0,...,w_d) = \left( \mathbb{C}^{d+1}- 0\right)/\sim
\]
with
\[
(z_0, ...,z_d)\sim (t^{w_0}z_0, ..., t^{w_d}z_d)
\]
for all $t\in \mathbb{C}^\times$.

Let $Y$ be a compact complex orbifold with only cyclic stabilizer groups. We should think of $\tilde{\Sigma}$.

\begin{definition}

A line bundle $L$ is \textbf{locally ample} if for all $y\in Y$, the stabilizer group of $Y$ acts faithfully on the fiber $L|_Y$.

\end{definition}

\begin{definition}

$L$ is \textbf{globally positive} if $L^N$ is a pullback of an ample line bundle from $Y$.

\end{definition}

\begin{definition}

$L$ is \textbf{orbi-ample} if it is locally ample and globally positive

\end{definition}

\begin{definition}

Let $n_i = |H^0(L^i)|$ for $i$. A \textbf{$k$-framing} of $L$ is a tuple
\[
\left(f_{ij}\right)_{i=k,...,2k, \,j=0,...,n}
\]
where $f_{ij}, J=0,...,n$ is a basis of $H^0(L^i)$.

\end{definition}

\begin{theorem}

Define $\mathbb{P}_k(L)=\mathbb{P}(k,...,k, k+1,....,k+1,..., 2k, ...,2k)$ and let $n_i$ be the number of $i$'s in this expression. Take $\phi_F: Y\to \mathbb{P}_k(L)$ to be the map that sends $y$ to $[\tau f_{ij}(y)]_{i=k,..., 2k, \, j=0,...,n_i}\subset \mathbb{P}_k(L)$. Then
\[
\tau: L|_y \to \mathbb{C}
\]
is an isomorphism.

\end{theorem}