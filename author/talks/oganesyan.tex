\chapter{Vardan Oganesyan: How to Construct Symplectic Homotopy Theory}
\label{oganesyan}

\abstract{In 1968 Dold and Thom proved that singular homology groups of X are isomorphic to homotopy groups of infinite symmetric product of X. In 1990-2000 Morel, Suslin, and Voevodsky used a similar definition to define motivic cohomology groups of algebraic varieties. Moreover, they defined homotopy theory for algebraic varieties. Motivated by these results, we construct homotopy theory for symplectic manifolds. In particular, we define some new homology groups for symplectic manifolds and prove that these homology groups have all required properties. We will not discuss details, but we will show that these new homology groups appear in a very natural way. If time permits, we will also discuss some possible applications.}

\section{Introduction}

Let $(X,e)$ where $e\in X$. Consider
\[
\text{SP}^n(X) = X^n/S_n
\]

We have an embedding $\text{SP}^n(X) \hookrightarrow \text{SP}^{n+1}(X)$ through $\{p_1,...,p_n\}\to \{p_1,...,p_n, e\}$. So we have $\text{SP}^0(X) \hookrightarrow \text{SP}^1(X) \hookrightarrow ...$

Define
\[
\text{SP}(X) =\bigcup \text{SP}^n(X)
\]
This consists of elements $p=\{ e, p_1,...,p_n, e\}$. It turns out $\text{SP}(X)$ is a semigroup under the following operation:
\[
\{ p_1,...,p_n, e\} + \{q_1,...,q_k, e\} = \{p_1,...,p_n, q_1,...,q_k, e\}
\]

Let $\Delta^n$ be an $n$-simplex. We consider
\[
\text{Map}(\Delta^n, \text{SP(X)})^+
\]
which is also a semigroup. We can turn this into a gorup by considering the Grothendieck group
\[
\text{Map}(\Delta^n, \text{SP(X)})^+=C_n(X)
\]
by adding $-f, -g$ satisfying $-(f+g)=-f-g$ and taking the restriction $\partial_K:C_n(X) \to C_{n-1}\to X$ with $d=\sum_{k=0}^n (-1)^k \partial_X$ satisfying $d^2=0$.

\section{Dold-Thom Theorems}

\begin{theorem}
[Dold, Thom, 1968]

\[
H(C_*(X); d)= H_*^{\text{sing}}(X; \mathbb{Z}).
\]
where $H(C_*(X); d)=\pi_*(\text{SP}(X))$

\end{theorem}

\begin{theorem}
[Dold, Thom, 1988-2000]

Let $X$ be a variety, $\text{SP}^n(X)$, $\Delta_{\text{alg}}^n=\{ z\in \mathbb{C}^{n-1} \mid z_1+...+z_{n+1} =1\}$, Consider $C_n^{\text{alg}}=\text{Map}_{\text{alg}}(\Delta_{\text{alg}}^n, \text{SP}(X))^+$. Define
\[
H(C_*^{\text{alg}, d} )=H_*^{\text{sus}}(X)
\]
where $\text{sus}$ means the Suslin homology.

We have
\[
H_0^{\text{sus}}(T^2) = \mathbb{Z}\times T^2 \\
H_0^{\text{sus}}(\Sigma_g)=\mathbb{Z}\times \text{Jac}(\Sigma_g)
\]

\end{theorem}

\section{Categories}

Let $Y$ be a variety, $\mathcal{U}\subset Y$ be open. We can define
\[
C_n(\mathcal{U}, X)=\text{Maps}_{\text{alg}} (\Delta_{\text{alg}}^n, \text{SP}(X))
\]
and $\mathcal{U}\to C_*(\mathcal{U}, X)$ is a sheaf on $Y$.

We can define the following categories:
\begin{enumerate}
\item Map are symplectic embeddings
\item Map are generalized Lagrangians corr
\item $J$-holomorphic maps
\end{enumerate}

The second one is the most important but also the hardest.

\begin{problem}
    What is $\Delta^n$? What is $\text{Map}(Y, \text{SP}(X))$? $\text{Map}(Y, \text{SP}(X))$?
\end{problem}

Consider $\mathcal{U} \stackrel{\text{symplectic}}{\hookrightarrow} X^n/S_n$. We are interested in $U\longrightarrow$ unordered maps $f_1,...,f_n: U\to X$ such that $f_1^* w_X + ... +f_n^* w_X= w_Y$ which is a presheaf. When we perform sheafification, we obtain a global section $\int \text{SCor}_n(Y,X)$. This is an analogue of maps $Y\to \text{SP}^n(X)$.

We need to define a map $Y\to \text{SP}(X)$. Consider $\text{ICor}(Y,X)$ in the same way: $\mathcal{U}\stackrel{\text{isotropic}}{\longrightarrow} X^n$ where if $G\in \text{ICor}_k(Y,X), F\in \text{SCor}_n(T, X)$, then $F+G\in \text{SCor}_{n+k}(T,X)$.

\begin{definition}

We say $F_1\in \text{SCor}_n(Y,X), F_2 \in \text{Scor}_{n+k}(Y,X)$, then $F_1\sim F_2$ if there exists $G\in \text{ICor}(T,X)$ such that $F_2=F_1+G$.

\end{definition}

We want to define $\text{SCor}(Y,X)= \left(\bigsqcup \text{SCor}_n(Y,X) \right)/\sim$ an analogue of $Y\to \text{SP}(X)$. Fix a segment $M=(M, w_M, p_0, p_1)$.

Now, we define $\text{SC}_n(Y,X) = \text{Map}(Y\times M^n, X)$. We have an embedding $i_k^\epsilon: M^n \hookrightarrow M^{n+1}$ where $M^n \to M^k\times p_\epsilon \times M^{n-k}$ where $\epsilon =0,1$. We have
\[
\partial_k: \text{SC}_n(T,X) \to \text{SC}_{n-1}(Y,X)\\
d=\sum_{\epsilon=0}^1 \sum_{k=0}^n \partial_k
\]
where $d^2=0$.

We obtain
\[
H(\text{SC}_*(Y,X); d)=\text{EH}_*(Y, X).
\]

All standard symplectic embeddings belong to $\text{SCor}(Y,X)$. If $Y$ is contractible, then $\text{SCor}(Y,X)\neq 0$.

\section{Homotopy and Triangulated Persistence Categories}

\begin{definition}

Let $F_0, F_1 \in \text{SCor}(Y,X)$. $F_0$ is \textbf{$M$-homotopic} to $F_1$ if there exists $H\in \text{SCor}(Y\times M, X)$ such that
\[
H|_{Y\times p_0}=F_0, \qquad H|_{Y\times p_2}=F_2.
\]

\end{definition}

\begin{proposition}
\text{ }
\begin{enumerate}
\item Homotopy equivalences defines equivalence relation $\text{SCor}(Y,X)$.
\item If $\varphi_t: Y\to X$ is a symplectic isotopy, then $\varphi_0$ is homotopic to $\varphi_1$.
\item Define $H(\text{SC}_*(Y,X); d) = \text{EH}_*(Y,X)$. Groups $\text{EH}_*(Y,X)$ are functorial in all nice ways.
\item These groups are homotopy invariant.
\item These groups have all of the required exact sequences.
\end{enumerate}

\end{proposition}

[Biran, Corea, Zhang] defined triangulated persistence category on $\mathcal{A}$, where $\text{ob}(\mathcal{A})$ are symplectic manifolds and $\text{Mor}(T,X)=\text{SCor}(T,X)$. This category is additive, and we can consider a category of chain complexes to get a triangulated persistence category with metric $\text{dist}(Y_1, Y_2)=\text{dist}_*(Y_1,X) \cdot \text{SC}_*(Y_2,X)$.

We can define $\text{JH}_*(T,X)$. Let $M=\mathbb{CP}^\times$. If $X$ is Kähler and $\text{JH}_0(\text{pt}, X)=0$, then $X$ is algebraic.