%%%%%%%%%%%%%%%%%%%%%part.tex%%%%%%%%%%%%%%%%%%%%%%%%%%%%%%%%%%
% 
% sample part title
%
% Use this file as a template for your own input.
%
%%%%%%%%%%%%%%%%%%%%%%%% Springer %%%%%%%%%%%%%%%%%%%%%%%%%%

\begin{partbacktext}
\part{Talks}

There were ten talks:\\

\begin{enumerate}
    \item \href{#pieloch}{Alex Pieloch: Spectral Equivalence of Nearby Lagrangians}
    
    Fix a commutative ring spectrum $R$. In this talk, we will show that any nearby Lagrangian in a cotangent bundle of a closed manifold is equivalent in the wrapped Fukaya category with $R$-coefficients to an $R$-brane supported on the zero section. As an application, we impose topological restrictions on the embeddings of exact Lagrangian fillings of the standard Legendrian unknot in sub-critical Stein domains. This is joint work with Johan Asplund and Yash Deshmukh.

    \item \href{#cristofaro}{Dan Cristofaro-Gardiner: Low-Action Holomorphic Curves and Invariant Sets}
    
    I will discuss a new compactness theorem for sequences of low-action punctured holomorphic curves of controlled topology, in any dimension, without imposing the typical assumption of uniformly bounded Hofer energy; in the limit, we extract a family of closed Reeb-invariant subsets. I will also explain why such sequences exist in abundance in low-dimensional symplectic dynamics, via the theory of embedded contact homology. This has various applications: the one I want to focus on in my talk is a generalization to higher genus surfaces and three-manifolds of the celebrated Le Calvez–Yoccoz theorem. All of this is joint with Rohil Prasad.

    \item \href{#pomerleano}{Daniel Pomerleano: Homological Mirror Symmetry for Batyrev Mirror Pairs}
    
    I will survey a recent proof of a version of Kontsevich’s homological mirror symmetry conjecture for a large class of mirror pairs of Calabi–Yau hypersurfaces in toric varieties. These mirror pairs were constructed by Batyrev from dual reflexive polytopes. The theorem holds in characteristic zero and in all but finitely many positive characteristics. This is joint work with Ganatra, Hanlon, Hicks, and Sheridan.

    \item \href{#wang}{Luya Wang: Deformation Inequivalent Symplectic Structures and Donaldson's Four-Six Question}

    Studying symplectic structures up to deformation equivalences is a fundamental question in symplectic geometry. Donaldson asked: given two homeomorphic closed symplectic four-manifolds, are they diffeomorphic if and only if their stabilized symplectic six-manifolds, obtained by taking products with $\mathbb{CP}^1$ with the standard symplectic form, are deformation equivalent? I will discuss joint work with Amanda Hirschi on showing how deformation inequivalent symplectic forms remain deformation inequivalent when stabilized, under certain algebraic conditions. This gives the first counterexamples to one direction of Donaldson’s “four-six” question and the related Stabilizing Conjecture by Ruan. In the other direction, I will also discuss more supporting evidence via Gromov–Witten invariants.

    \item \href{#hendricks}{Kristen Hendricks: Symplectic Annular Khovanov Homology and Knot Symmetry}
    
    Khovanov homology is a combinatorially-defined invariant which has proved to contain a wealth of geometric information. In 2006 Seidel and Smith introduced a candidate Lagrangian Floer analog of the theory, which has been shown by Abouzaid and Smith to be isomorphic to the original theory over fields of characteristic zero. The relationship between the theories is still unknown over other fields. In 2010 Seidel and Smith showed there is a spectral sequence relating the symplectic Khovanov homology of a two-periodic knot to the symplectic Khovanov homology of its quotient; in contrast, in 2018 Stoffregen and Zhang used the Khovanov homotopy type to show that there is a spectral sequence from the combinatorial Khovanov homology of a two-periodic knot to the annular Khovanov homology of its quotient. (An alternate proof of this result was subsequently given by Borodzik, Politarczyk, and Silvero.) These results necessarily use coefficients in the field of two elements. This inspired investigations of Mak and Seidel into an annular version of symplectic Khovanov homology, which they defined over characteristic zero. In this talk we introduce a new, conceptually straightforward, formulation of symplectic annular Khovanov homology, defined over any field. Using this theory, we show how to recover the Stoffregen-Zhang spectral sequence on the symplectic side. We further give an analog of recent results of Lipshitz and Sarkar for the Khovanov homology of strongly invertible knots. This is work in progress with Cheuk Yu Mak and Sriram Raghunath.

    \item \href{#pardon}{John Pardon: Derived Moduli Spaces of Pseudo-Holomorphic Curves}
    
    I will present the derived representability approach to working with moduli spaces of pseudo-holomorphic curves.

    \item \href{#mclean}{Mark McLean: Symplectic Orbifold Gromov-Witten Invariants}
    
    Chen and Ruan constructed symplectic orbifold Gromov-Witten invariants more than 20 years ago. In ongoing work with Alex Ritter, we show that moduli spaces of pseudo-holomorphic curves mapping to a symplectic orbifold admit global Kuranishi charts. This allows us to construct other types of Gromov-Witten invariants, such as K-theoretic counts. The construction relies on an orbifold embedding theorem of Ross and Thomas.

    \item \href{#prasad}{Rohil Prasad: High-Dimensional Families of Holomorphic Curves and Three-Dimensional Energy Surfaces}
    
    Let $H$ be any smooth function on $\mathbb{R}^4$. I’ll discuss some recent dynamical theorems for the Hamiltonian flow on level sets of H (“energy surfaces”). The results are proved using holomorphic curves and neck stretching. One important tool is the compactness theorem from Dan’s talk.

    \item \href{#massoni}{Thomas Massoni: Taut Foliations Through a Contact Lens}
    
    In the late ’90s, Eliashberg and Thurston established a remarkable connection between foliations and contact structures in dimension three: any co-oriented, aspherical foliation on a closed, oriented 3-manifold can be approximated by both positive and negative contact structures. Additionally, if the foliation is taut then its contact approximations are tight. In this talk, I will present a converse result on constructing taut foliations from suitable pairs of contact structures. While taut foliations are rather rigid objects, this viewpoint reveals some degree of flexibility and offers a new perspective on the $L$-space conjecture.

    \item \href{#oganesyan}{Vardan Oganesyan: How to Construct Symplectic Homotopy Theory}
    
    In 1968 Dold and Thom proved that singular homology groups of X are isomorphic to homotopy groups of infinite symmetric product of X. In 1990-2000 Morel, Suslin, and Voevodsky used a similar definition to define motivic cohomology groups of algebraic varieties. Moreover, they defined homotopy theory for algebraic varieties. Motivated by these results, we construct homotopy theory for symplectic manifolds. In particular, we define some new homology groups for symplectic manifolds and prove that these homology groups have all required properties. We will not discuss details, but we will show that these new homology groups appear in a very natural way. If time permits, we will also discuss some possible applications.
\end{enumerate}

\end{partbacktext}