\chapter{Daniel Pomerleano: Homological Mirror Symmetry for Batyrev Mirror Pairs}
\label{pomerleano}

\abstract{I will survey a recent proof of a version of Kontsevich’s homological mirror symmetry conjecture for a large class of mirror pairs of Calabi–Yau hypersurfaces in toric varieties. These mirror pairs were constructed by Batyrev from dual reflexive polytopes. The theorem holds in characteristic zero and in all but finitely many positive characteristics. This is joint work with Ganatra, Hanlon, Hicks, and Sheridan.}

\section{Set Up}

Let $K$ be a field and $M_R = M\otimes_{\mathbb{Z}} \mathbb{R}$. Fix a lattice $M\cong Z^n$ with $n\ge 4$ and a reflexive polytope $\Delta$ in $M_{\mathbb{R}}$. Let $\Delta_* \subset M_{\mathbb{R}}^*$ be the dual polytope and $\overline{\Sigma} \subset M_{\mathbb{R}}^*$ be the fan dual to $\Delta$ (rays of the fan point along the vertices of $\Delta^*$)

We will assume that $\Sigma^*$, the fan dual to $\Delta^*$, is a smooth fan. Additionally, assume that $\overline{\Sigma} \rightsquigarrow \overline{Y}$ is a toric variety where $\Sigma^* \rightsquigarrow Y^*$ smooth. Let $P$ denote the integer lattice points $\Delta^* \cap M^*$ and let $P\subset P^0$ be the subset of lattice points which lie on a face of codimension $\ge 2$.

\section{B-Side}
Consider $\mathcal{L}_{\Delta^*} \to Y^*$ and let
\[
W_r = - z^0 + \sum_{p\in P} r_pz^p
\]

Then $(r_p) \in \wedge_K^P$ where $\wedge_K:= \sum_{i=0}^\infty a_iT^{b_i}$ satisfying $\lim_{i\to \infty} b_i = \infty, a_i \in K$ is the Novikov ring. We have
\[
X_r^* = \{ W_r =0\}\subset Y.
\]

\section{A-Side}

Assume $\overline{Y}$ is not assumed to be smooth, $A=\Delta\cap M$, and $z^\alpha \in \Gamma(\overline{Y}, \mathcal{L}_\Delta)$. Let $\Sigma$ be a refinement of $\overline{\Sigma}$, where $\Sigma(1)=P$. Assume $Y=Y_\Sigma$ is smooth away from dimension
    \[
        \overline{X}_t = \{ -tz^0 + \sum_{\alpha \in A\backslash 0} z^\alpha =0 \}\subset \overline{Y}
    \]
The proper transform $X_t$ is a smooth Calabi-Yau in $Y$.

Consider a Kähler class on $Y$ of the form
\[
[w]=\sum_{p\in P} \ell_p \text{PD}([D_P^Y]), \qquad \ell_p \in \mathbb{R}^{>0}.
\]
and restrict it to $X_t$.

On the A-side, we consider a variant of the Fukaya category
\[
\text{Fut}(X_t, D; \wedge)
\]
where objects of this category are compact exact Lagrangian submanifolds in $X_t \backslash D$ and holomorphic curves $u$ are weighted by $T^{\Sigma \ell_p u \cdot D_p}$.

\begin{theorem}

Suppose that $D_p$ are connected. Away from finitely many bad characteristics, there exist $b(\wedge)=(b(\wedge))
_{p\in P} \in \wedge^P$ with $\text{val}(b(\wedge)_p)=\ell_p$
and an equivalence
\[
\text{Fuk}(X_t, D; A)\cong \mathcal{D}^b\text{Coh}(X_{b(\wedge)})^*
\]

\end{theorem}

\begin{remark}

In characteristic 0, homological mirror symmetry implies Givental's Hodge-theoretic mirror symmetry.

\end{remark}

This motivates the following problem:

\begin{problem}

Is there some kind of Gromov-Witten implication of homological mirror symmetry in a positive characteristic?

\end{problem}

\section{Strategy of Proof}

This follows the groundbreaking worth of Seidel (in the case of quartic surface in $\mathbb{P}^3$):

\begin{enumerate}
\item Step 1: $\text{Fuk}(X_t\backslash D) \cong \mathcal{D}^b\text{Coh}(\partial Y^*)$ where $\partial Y^*$ is the toric divisor which is cut out by $z^0$. For $\mathcal{A}_0 \subset \text{Fuk}(X_t\backslash D)$, $\mathcal{B}_\gamma:= \{\theta(i)\}_{i\in \mathbb{Z}}$.
\item Step 2: Employ a deformation theory argument.
\end{enumerate}

We will only talk about Step 1.

Let $H=X_t\backslash D \subset (\mathbb{C}^\times)^n$. We can consider $\mathcal{W}H)$ where we allow certain non-compact Lagrangians and $\mathcal{W}(\mathbb{C}^\times)^n, H)$.

\begin{theorem}
[Gammage, Shende]

\[\begin{tikzcd}[column sep=1em]
	{\mathcal{W}H)} && {\mathcal{D}^b\text{Coh}(\partial Y^*)} \\
	{\mathcal{W}(\mathbb{C}^\times)^n, H)} && {\mathcal{D}^b\text{Coh}(Y^*)}
	\arrow["\cong"{description}, draw=none, from=1-1, to=1-3]
	\arrow["\cup", from=1-1, to=2-1]
	\arrow[from=1-3, to=2-3]
	\arrow["\cong"{description}, draw=none, from=2-1, to=2-3]
\end{tikzcd}\]

\end{theorem}

Abouzaid considered a different form of homological mirror symmetry for toric varieties where he considers certain Lagrangian actions with boundary on this hypersurface $H$:
\[
\mathcal{F}_{\text{trop}}((\mathbb{C}^\times)^n, H) \simeq \text{Pic}^{dg}{Y^*}
\]

\begin{theorem}

\[\begin{tikzcd}
	{\mathcal{W}(H)} && {\mathcal{D}\text{Coh}(\partial Y^\times)} \\
	\\
	{\mathcal{W}\left(\left(C^\times\right)^n, H\right)} && {\mathcal{D}^b\text{Coh}(Y^\times)} \\
	{\mathcal{F}_{\text{trop}}\left(\left(\mathbb{C}^\times \right)^n, H\right)} && {\text{Pic}^{\text{dg}}(Y^\times)}
	\arrow["\cong"', from=1-1, to=1-3]
	\arrow["{\text{GS}}", from=1-1, to=1-3]
	\arrow["{(\cup)^*}"', from=3-1, to=1-1]
	\arrow["{\text{KGPS}}", from=3-1, to=3-3]
	\arrow["\cong"', from=3-1, to=3-3]
	\arrow["{C^\times}"', from=3-3, to=1-3]
	\arrow["{\subset\text{ with the fiber}}", curve={height=-50pt}, from=4-1, to=1-1]
	\arrow["\cong"{description}, draw=none, from=4-1, to=4-3]
	\arrow["A", draw=none, from=4-1, to=4-3]
\end{tikzcd}\]

\end{theorem}

One thing that gets used that wasn't available to Seidel/Sheridan is that
\[
\text{CO}: \text{SH}^*(X_t\backslash D) \longrightarrow \text{HH}^*(\mathcal{W}X_t\backslash D))\longrightarrow \text{HH}^*(\text{Fut}(X_t\backslash D))
\]
are isomorphisms and $\text{SH}^*(X_t\backslash D)$ can be computed in terms of the topology of the pair $(X_t,D)$.